% Options for packages loaded elsewhere
\PassOptionsToPackage{unicode}{hyperref}
\PassOptionsToPackage{hyphens}{url}
\PassOptionsToPackage{dvipsnames,svgnames,x11names}{xcolor}
%
\documentclass[
  12pt,
]{article}
\usepackage{amsmath,amssymb}
\usepackage{lmodern}
\usepackage{iftex}
\ifPDFTeX
  \usepackage[T1]{fontenc}
  \usepackage[utf8]{inputenc}
  \usepackage{textcomp} % provide euro and other symbols
\else % if luatex or xetex
  \usepackage{unicode-math}
  \defaultfontfeatures{Scale=MatchLowercase}
  \defaultfontfeatures[\rmfamily]{Ligatures=TeX,Scale=1}
\fi
% Use upquote if available, for straight quotes in verbatim environments
\IfFileExists{upquote.sty}{\usepackage{upquote}}{}
\IfFileExists{microtype.sty}{% use microtype if available
  \usepackage[]{microtype}
  \UseMicrotypeSet[protrusion]{basicmath} % disable protrusion for tt fonts
}{}
\makeatletter
\@ifundefined{KOMAClassName}{% if non-KOMA class
  \IfFileExists{parskip.sty}{%
    \usepackage{parskip}
  }{% else
    \setlength{\parindent}{0pt}
    \setlength{\parskip}{6pt plus 2pt minus 1pt}}
}{% if KOMA class
  \KOMAoptions{parskip=half}}
\makeatother
\usepackage{xcolor}
\usepackage[margin=1in]{geometry}
\usepackage{color}
\usepackage{fancyvrb}
\newcommand{\VerbBar}{|}
\newcommand{\VERB}{\Verb[commandchars=\\\{\}]}
\DefineVerbatimEnvironment{Highlighting}{Verbatim}{commandchars=\\\{\}}
% Add ',fontsize=\small' for more characters per line
\usepackage{framed}
\definecolor{shadecolor}{RGB}{248,248,248}
\newenvironment{Shaded}{\begin{snugshade}}{\end{snugshade}}
\newcommand{\AlertTok}[1]{\textcolor[rgb]{0.94,0.16,0.16}{#1}}
\newcommand{\AnnotationTok}[1]{\textcolor[rgb]{0.56,0.35,0.01}{\textbf{\textit{#1}}}}
\newcommand{\AttributeTok}[1]{\textcolor[rgb]{0.77,0.63,0.00}{#1}}
\newcommand{\BaseNTok}[1]{\textcolor[rgb]{0.00,0.00,0.81}{#1}}
\newcommand{\BuiltInTok}[1]{#1}
\newcommand{\CharTok}[1]{\textcolor[rgb]{0.31,0.60,0.02}{#1}}
\newcommand{\CommentTok}[1]{\textcolor[rgb]{0.56,0.35,0.01}{\textit{#1}}}
\newcommand{\CommentVarTok}[1]{\textcolor[rgb]{0.56,0.35,0.01}{\textbf{\textit{#1}}}}
\newcommand{\ConstantTok}[1]{\textcolor[rgb]{0.00,0.00,0.00}{#1}}
\newcommand{\ControlFlowTok}[1]{\textcolor[rgb]{0.13,0.29,0.53}{\textbf{#1}}}
\newcommand{\DataTypeTok}[1]{\textcolor[rgb]{0.13,0.29,0.53}{#1}}
\newcommand{\DecValTok}[1]{\textcolor[rgb]{0.00,0.00,0.81}{#1}}
\newcommand{\DocumentationTok}[1]{\textcolor[rgb]{0.56,0.35,0.01}{\textbf{\textit{#1}}}}
\newcommand{\ErrorTok}[1]{\textcolor[rgb]{0.64,0.00,0.00}{\textbf{#1}}}
\newcommand{\ExtensionTok}[1]{#1}
\newcommand{\FloatTok}[1]{\textcolor[rgb]{0.00,0.00,0.81}{#1}}
\newcommand{\FunctionTok}[1]{\textcolor[rgb]{0.00,0.00,0.00}{#1}}
\newcommand{\ImportTok}[1]{#1}
\newcommand{\InformationTok}[1]{\textcolor[rgb]{0.56,0.35,0.01}{\textbf{\textit{#1}}}}
\newcommand{\KeywordTok}[1]{\textcolor[rgb]{0.13,0.29,0.53}{\textbf{#1}}}
\newcommand{\NormalTok}[1]{#1}
\newcommand{\OperatorTok}[1]{\textcolor[rgb]{0.81,0.36,0.00}{\textbf{#1}}}
\newcommand{\OtherTok}[1]{\textcolor[rgb]{0.56,0.35,0.01}{#1}}
\newcommand{\PreprocessorTok}[1]{\textcolor[rgb]{0.56,0.35,0.01}{\textit{#1}}}
\newcommand{\RegionMarkerTok}[1]{#1}
\newcommand{\SpecialCharTok}[1]{\textcolor[rgb]{0.00,0.00,0.00}{#1}}
\newcommand{\SpecialStringTok}[1]{\textcolor[rgb]{0.31,0.60,0.02}{#1}}
\newcommand{\StringTok}[1]{\textcolor[rgb]{0.31,0.60,0.02}{#1}}
\newcommand{\VariableTok}[1]{\textcolor[rgb]{0.00,0.00,0.00}{#1}}
\newcommand{\VerbatimStringTok}[1]{\textcolor[rgb]{0.31,0.60,0.02}{#1}}
\newcommand{\WarningTok}[1]{\textcolor[rgb]{0.56,0.35,0.01}{\textbf{\textit{#1}}}}
\usepackage{longtable,booktabs,array}
\usepackage{calc} % for calculating minipage widths
% Correct order of tables after \paragraph or \subparagraph
\usepackage{etoolbox}
\makeatletter
\patchcmd\longtable{\par}{\if@noskipsec\mbox{}\fi\par}{}{}
\makeatother
% Allow footnotes in longtable head/foot
\IfFileExists{footnotehyper.sty}{\usepackage{footnotehyper}}{\usepackage{footnote}}
\makesavenoteenv{longtable}
\usepackage{graphicx}
\makeatletter
\def\maxwidth{\ifdim\Gin@nat@width>\linewidth\linewidth\else\Gin@nat@width\fi}
\def\maxheight{\ifdim\Gin@nat@height>\textheight\textheight\else\Gin@nat@height\fi}
\makeatother
% Scale images if necessary, so that they will not overflow the page
% margins by default, and it is still possible to overwrite the defaults
% using explicit options in \includegraphics[width, height, ...]{}
\setkeys{Gin}{width=\maxwidth,height=\maxheight,keepaspectratio}
% Set default figure placement to htbp
\makeatletter
\def\fps@figure{htbp}
\makeatother
\setlength{\emergencystretch}{3em} % prevent overfull lines
\providecommand{\tightlist}{%
  \setlength{\itemsep}{0pt}\setlength{\parskip}{0pt}}
\setcounter{secnumdepth}{5}
\newlength{\cslhangindent}
\setlength{\cslhangindent}{1.5em}
\newlength{\csllabelwidth}
\setlength{\csllabelwidth}{3em}
\newlength{\cslentryspacingunit} % times entry-spacing
\setlength{\cslentryspacingunit}{\parskip}
\newenvironment{CSLReferences}[2] % #1 hanging-ident, #2 entry spacing
 {% don't indent paragraphs
  \setlength{\parindent}{0pt}
  % turn on hanging indent if param 1 is 1
  \ifodd #1
  \let\oldpar\par
  \def\par{\hangindent=\cslhangindent\oldpar}
  \fi
  % set entry spacing
  \setlength{\parskip}{#2\cslentryspacingunit}
 }%
 {}
\usepackage{calc}
\newcommand{\CSLBlock}[1]{#1\hfill\break}
\newcommand{\CSLLeftMargin}[1]{\parbox[t]{\csllabelwidth}{#1}}
\newcommand{\CSLRightInline}[1]{\parbox[t]{\linewidth - \csllabelwidth}{#1}\break}
\newcommand{\CSLIndent}[1]{\hspace{\cslhangindent}#1}
\usepackage{polyglossia}
\setmainlanguage{turkish}
\usepackage{booktabs}
\usepackage{caption}
\captionsetup[table]{skip=10pt}
\ifLuaTeX
  \usepackage{selnolig}  % disable illegal ligatures
\fi
\IfFileExists{bookmark.sty}{\usepackage{bookmark}}{\usepackage{hyperref}}
\IfFileExists{xurl.sty}{\usepackage{xurl}}{} % add URL line breaks if available
\urlstyle{same} % disable monospaced font for URLs
\hypersetup{
  pdftitle={Çalışmanızın Başlığı},
  pdfauthor={İsim Soyisim},
  colorlinks=true,
  linkcolor={Maroon},
  filecolor={Maroon},
  citecolor={Blue},
  urlcolor={blue},
  pdfcreator={LaTeX via pandoc}}

\title{Çalışmanızın Başlığı}
\author{İsim Soyisim\footnote{Öğrenci Numarası, \href{https://github.com/KULLANICI_ADINIZ/REPO_ADINIZ}{Github Repo}}}
\date{}

\begin{document}
\maketitle
\begin{abstract}
Bu bölümde çalışmanızın özetini yazınız.
\end{abstract}

\hypertarget{final-hakkux131nda-uxf6nemli-bilgiler}{%
\section{Final Hakkında Önemli Bilgiler}\label{final-hakkux131nda-uxf6nemli-bilgiler}}

\colorbox{BurntOrange}{GITHUB REPO BAĞLANTINIZI BU DOSYANIN 37. SATIRINA YAZINIZ!}

\textbf{Proje gönderimi, Github repo linki ile birlikte ekampus sistemine bir zip dosyası yüklenerek yapılacaktır. Sisteme zip dosyası yüklemezseniz ve Github repo linki vermezseniz ara sınav ve final sınavlarına girmemiş sayılırsınız.}

\textbf{Proje klasörünüzü sıkıştırdıktan sonra (\texttt{OgrenciNumarasi.zip} dosyası) 9 Haziran 2023 23:59'a kadar \emph{ekampus.ankara.edu.tr} adresine yüklemeniz gerekmektedir.}

\colorbox{WildStrawberry}{Daha fazla bilgi için proje klasöründeki README.md dosyasını okuyunuz.}

\hypertarget{giriux15f}{%
\section{Giriş}\label{giriux15f}}

Bu taslak size proje ödevinizi yazarken yardımcı olması amacıyla oluşturulmuştur. Ödevlerinizde, makalelerinizde, sunumlarınızda ve projelerinizde kullandığınız tüm bilgi kaynaklarına atıfta bulunmalısınız. Alıntı ve gönderme yapmak okuyuculara çalışmanızda kullandığınız/başvurduğunuz kaynaklara ulaşma imkanı sağlar. \textbf{Her ne kadar kendi sözlerinizi kullanıyor olsanız da, başkalarına ait fikirleri çalışmanızda aktarıyorsanız bu fikirlerin kaynağını belgelemek zorundasınız. Aksi takdirde akademik intihal yapmış olursunuz.} Yazım konusunda (\protect\hyperlink{ref-aydinonat:2007}{\textbf{aydinonat:2007?}})'ye başvurabilirsiniz.

Proje ödevinizde yer alan başlıkların bu metinde yer alan başlıkları kesinlikle içermesi gerekmektedir. Burada kullanılan başlıklar haricinde farklı alt başlıklar da kullanabilirsiniz. Projenizi yazarken bu dosyayı taslak olarak kullanıp içeriğini projenize uyarlayınız.

\hypertarget{uxe7alux131ux15fmanux131n-amacux131}{%
\subsection{Çalışmanın Amacı}\label{uxe7alux131ux15fmanux131n-amacux131}}

Bu bölümde yaptığınız çalışmanın amacından ve öneminden birkaç paragraf ile bahsediniz.

\hypertarget{literatuxfcr}{%
\subsection{Literatür}\label{literatuxfcr}}

Bu bölümde konu ile ilgili olarak okuduğunuz makaleleri referans vererek tartışınız. \textbf{Her makaleyi ayrı başlık altında tek tek özetlemeyiniz.} Literatür taramasında \textbf{en az altı} makaleye atıf yapılması ve bu makalelerden \textbf{en az ikisinin İngilizce} olması gerekmektedir.

\hypertarget{veri}{%
\section{Veri}\label{veri}}

Bu bölümde çalışmanızda kullandığınız veri setinin kaynağını, ham veri üzerinde herhangi bir işlem yaptıysanız bu işlemleri ve veri seti ile ilgili özet istatistikleri tartışınız. Bu bölümde tüm değişkenlere ait özet istatistikleri (ortalama, standart sapma, minimum, maksimum, vb. değerleri) içeren bir tablo (Tablo \ref{tab:ozet}) olması zorunludur. Tablolarınıza gerekli göndermeleri bir önceki cümlede gösterildiği gibi yapınız. (\protect\hyperlink{ref-perkins:1991}{\textbf{perkins:1991?}})

Analize ait R kodları bu bölümde başlamalıdır. Bu bölümde veri setini R'a aktaran ve özet istatistikleri üreten kodlar yer almalıdır.

\begin{Shaded}
\begin{Highlighting}[]
\FunctionTok{library}\NormalTok{(tidyverse)}
\FunctionTok{library}\NormalTok{(here)}
\NormalTok{data }\OtherTok{\textless{}{-}} \FunctionTok{read\_csv}\NormalTok{(}\FunctionTok{here}\NormalTok{(}\StringTok{"../data/istih.csv"}\NormalTok{))}
\FunctionTok{View}\NormalTok{(data)}
\end{Highlighting}
\end{Shaded}

Rmd dosyasında kod bloklarının bazılarında kod seçeneklerinin düzenlendiğine dikkat edin.

\texttt{echo=FALSE} seçeneği ile kodların türetilen pdf dosyasında görünmesini engelleyin ve sonuçlarınızı tablo halinde rapor edin.

\begin{table}[ht]
\centering
\caption{Özet İstatistikler} 
\label{tab:ozet}
\begin{tabular}{1ccccc}
  \toprule
 & Ortalama & Std.Sap & Min & Medyan & Mak \\ 
  \midrule
Value & 15.49 & 5.93 & 6.64 & 14.19 & 34.88 \\ 
   \bottomrule
\end{tabular}
\end{table}

\begin{Shaded}
\begin{Highlighting}[]
\NormalTok{data }\SpecialCharTok{\%\textgreater{}\%}
  \FunctionTok{select}\NormalTok{(TIME,Value)}\SpecialCharTok{\%\textgreater{}\%}
  \FunctionTok{table}\NormalTok{()}
\end{Highlighting}
\end{Shaded}

\begin{verbatim}
##       Value
## TIME   6.6372008 7.435945 7.5818911 7.6713614 8.0662355 8.1621475 8.3903246
##   2018         0        0         1         1         0         0         0
##   2019         1        1         0         0         1         0         0
##   2020         0        0         0         0         0         1         1
##       Value
## TIME   8.6841755 8.7435341 8.8197355 8.8471212 8.9284897 8.9781923 9.584341
##   2018         0         0         1         0         0         0        0
##   2019         1         1         0         1         1         1        0
##   2020         0         0         0         0         0         0        1
##       Value
## TIME   10.001472 10.021872 10.023391 10.061987 10.264264 10.317203 10.387557
##   2018         1         1         0         1         1         0         1
##   2019         0         0         0         0         0         0         0
##   2020         0         0         1         0         0         1         0
##       Value
## TIME   10.586082 10.77473 10.883707 11.035892 11.640663 11.769904 11.860219
##   2018         0        0         0         0         0         0         1
##   2019         0        1         1         1         0         1         0
##   2020         1        0         0         0         1         0         0
##       Value
## TIME   12.027833 12.211666 12.415541 12.456524 12.552653 12.666578 12.712753
##   2018         1         1         1         1         0         0         0
##   2019         0         0         0         0         1         0         1
##   2020         0         0         0         0         0         1         0
##       Value
## TIME   12.903226 12.954085 13.406408 13.482139 13.551468 13.591121 13.714321
##   2018         0         0         1         1         1         0         0
##   2019         1         1         0         0         0         1         0
##   2020         0         0         0         0         0         0         1
##       Value
## TIME   13.771465 13.795696 13.82 13.871192 13.904705 14.003207 14.020249
##   2018         0         0     0         0         0         0         1
##   2019         0         0     0         1         1         1         0
##   2020         1         1     1         0         0         0         0
##       Value
## TIME   14.038448 14.038699 14.173528 14.194826 14.195583 14.473471 14.486988
##   2018         0         0         0         1         1         0         0
##   2019         0         0         0         0         0         1         1
##   2020         1         1         1         0         0         0         0
##       Value
## TIME   14.48907 14.584582 14.60678 14.64871 14.667459 14.674225 14.683599
##   2018        1         1        1        1         0         0         1
##   2019        0         0        0        0         0         0         0
##   2020        0         0        0        0         1         1         0
##       Value
## TIME   14.97879 15.231788 15.333426 15.365002 15.406888 15.427958 15.442274
##   2018        0         0         0         0         0         1         1
##   2019        1         0         1         0         1         0         0
##   2020        0         1         0         1         0         0         0
##       Value
## TIME   15.490126 15.541712 15.5643 15.876476 16.097354 16.77173 16.786358
##   2018         0         0       0         0         0        0         1
##   2019         1         0       0         0         0        0         0
##   2020         0         1       1         1         1        1         0
##       Value
## TIME   17.022766 17.416395 17.616419 18.177565 18.742828 18.991467 19.492943
##   2018         1         0         0         0         0         0         0
##   2019         0         0         0         1         0         0         1
##   2020         0         1         1         0         1         1         0
##       Value
## TIME   20.306179 21.085546 21.72349 21.903513 22.014969 22.047745 22.852537
##   2018         1         0        0         0         1         0         1
##   2019         0         0        0         1         0         1         0
##   2020         0         1        1         0         0         0         0
##       Value
## TIME   22.944685 23.034515 23.15937 24.805059 24.943991 26.108511 26.30431
##   2018         0         0        1         0         1         0        1
##   2019         1         1        0         0         0         0        0
##   2020         0         0        0         1         0         1        0
##       Value
## TIME   27.425533 27.456076 28.283504 28.457134 31.21693 33.295712 34.88187
##   2018         0         0         1         0        1         0        0
##   2019         0         1         0         1        0         1        0
##   2020         1         0         0         0        0         0        1
\end{verbatim}

\hypertarget{yuxf6ntem-ve-veri-analizi}{%
\section{Yöntem ve Veri Analizi}\label{yuxf6ntem-ve-veri-analizi}}

Bu bölümde veri setindeki bilgileri kullanarak çalışmanın amacına ulaşmak için kullanılacak yöntemleri açıklayın. Derste işlenen/işlenecek olan analiz yöntemlerinden (Hipotez testleri ve korelasyon analizi gibi) çalışmanın amacına ve veri setine uygun olanlar bu bölümde kullanılmalıdır. (\protect\hyperlink{ref-newbold:2003}{\textbf{newbold:2003?}}; \protect\hyperlink{ref-ozsoy:2010}{\textbf{ozsoy:2010?}}; \protect\hyperlink{ref-ozsoy:2014}{\textbf{ozsoy:2014?}})

\begin{Shaded}
\begin{Highlighting}[]
\FunctionTok{library}\NormalTok{(broom)}

\NormalTok{data }\SpecialCharTok{\%\textgreater{}\%} 
  \FunctionTok{group\_by}\NormalTok{(LOCATION) }\SpecialCharTok{\%\textgreater{}\%} 
  \FunctionTok{summarise}\NormalTok{(}\AttributeTok{var =} \FunctionTok{var}\NormalTok{(Value))}
\end{Highlighting}
\end{Shaded}

\begin{verbatim}
## # A tibble: 37 x 2
##    LOCATION    var
##    <chr>     <dbl>
##  1 AUS       6.71 
##  2 AUT       2.16 
##  3 BEL       0.473
##  4 CAN       0.556
##  5 CHE       1.10 
##  6 CHL      NA    
##  7 COL      21.7  
##  8 CRI       3.72 
##  9 CZE       0.171
## 10 DEU       1.19 
## # i 27 more rows
\end{verbatim}

\begin{Shaded}
\begin{Highlighting}[]
\FunctionTok{shapiro.test}\NormalTok{(data}\SpecialCharTok{$}\NormalTok{Value)}
\end{Highlighting}
\end{Shaded}

\begin{verbatim}
## 
##  Shapiro-Wilk normality test
## 
## data:  data$Value
## W = 0.90329, p-value = 1.228e-06
\end{verbatim}

\begin{Shaded}
\begin{Highlighting}[]
\FunctionTok{ggplot}\NormalTok{(data) }\SpecialCharTok{+}
  \FunctionTok{aes}\NormalTok{(}\AttributeTok{x =}\NormalTok{ Value, }\AttributeTok{fill =}\NormalTok{ LOCATION) }\SpecialCharTok{+}
  \FunctionTok{geom\_histogram}\NormalTok{(}\AttributeTok{binwidth =}\NormalTok{ .}\DecValTok{5}\NormalTok{, }\AttributeTok{alpha =} \FloatTok{0.5}\NormalTok{)}
\end{Highlighting}
\end{Shaded}

\includegraphics{final_files/figure-latex/unnamed-chunk-5-1.pdf}

Örneğin, regresyon analizi gerçekleştiriyorsanız tahmin ettiğiniz denklemi bu bölümde tartışınız. Denklemlerinizi ve matematiksel ifadeleri \(LaTeX\) kullanarak yazınız.

\[
Y_t = \beta_0 + \beta_N N_t + \beta_P P_t + \beta_I I_t + \varepsilon_t
\]

Bu bölümde analize ilişkin farklı tablolar ve grafiklere yer verilmelidir. Çalışmanıza uygun biçimde histogram, nokta grafiği (Şekil \ref{fig:plot} gibi), kutu grafiği, vb. grafikleri bu bölüme ekleyiniz. Şekillerinize de gerekli göndermeleri bir önceki cümlede gösterildiği gibi yapınız.

\begin{Shaded}
\begin{Highlighting}[]
\NormalTok{data }\SpecialCharTok{\%\textgreater{}\%} 
  \FunctionTok{ggplot}\NormalTok{(}\FunctionTok{aes}\NormalTok{(}\AttributeTok{x =}\NormalTok{ TIME, }\AttributeTok{y =}\NormalTok{ Value)) }\SpecialCharTok{+}
  \FunctionTok{geom\_point}\NormalTok{() }\SpecialCharTok{+}
  \FunctionTok{geom\_smooth}\NormalTok{() }\SpecialCharTok{+}
  \FunctionTok{scale\_x\_continuous}\NormalTok{(}\StringTok{"TIME"}\NormalTok{) }\SpecialCharTok{+} 
  \FunctionTok{scale\_y\_continuous}\NormalTok{(}\StringTok{"Value"}\NormalTok{)}
\end{Highlighting}
\end{Shaded}

\begin{figure}

{\centering \includegraphics{final_files/figure-latex/plot-1} 

}

\caption{Muhteşem Bir Grafik}\label{fig:plot}
\end{figure}

\begin{Shaded}
\begin{Highlighting}[]
\FunctionTok{ggplot}\NormalTok{(data, }\FunctionTok{aes}\NormalTok{(}\AttributeTok{x =}\NormalTok{ Value)) }\SpecialCharTok{+}
  \FunctionTok{geom\_histogram}\NormalTok{(}\AttributeTok{color=}\StringTok{"white"}\NormalTok{,}
                \AttributeTok{fill=} \StringTok{"black"}\NormalTok{ ) }\SpecialCharTok{+}
  \FunctionTok{scale\_x\_continuous}\NormalTok{(}\AttributeTok{name =} \StringTok{"İstihdam"}\NormalTok{) }\SpecialCharTok{+}
  \FunctionTok{scale\_y\_continuous}\NormalTok{(}\AttributeTok{name =} \StringTok{""}\NormalTok{) }\SpecialCharTok{+}
  \FunctionTok{ggtitle}\NormalTok{(}\StringTok{"İstihdam Oranı"}\NormalTok{)}
\end{Highlighting}
\end{Shaded}

\includegraphics{final_files/figure-latex/unnamed-chunk-6-1.pdf}

\begin{Shaded}
\begin{Highlighting}[]
\NormalTok{data }\SpecialCharTok{\%\textgreater{}\%}
  \FunctionTok{ggplot}\NormalTok{() }\SpecialCharTok{+}
  \FunctionTok{aes}\NormalTok{(}\AttributeTok{x =}\NormalTok{ Value, }\AttributeTok{group =}\NormalTok{ LOCATION, }\AttributeTok{fill =}\NormalTok{ LOCATION) }\SpecialCharTok{+}
  \FunctionTok{geom\_density}\NormalTok{() }\SpecialCharTok{+}
  \FunctionTok{facet\_wrap}\NormalTok{(}\SpecialCharTok{\textasciitilde{}}\NormalTok{LOCATION) }\SpecialCharTok{+}
  \FunctionTok{labs}\NormalTok{(}\AttributeTok{y =} \StringTok{""}\NormalTok{, }\AttributeTok{x =} \StringTok{""}\NormalTok{, }\AttributeTok{title =} \StringTok{"ÜLKELERE GÖRE İSTİHDAM ORANI"}\NormalTok{)}
\end{Highlighting}
\end{Shaded}

\includegraphics{final_files/figure-latex/unnamed-chunk-7-1.pdf}

\hypertarget{sonuuxe7}{%
\section{Sonuç}\label{sonuuxe7}}

Bu bölümde çalışmanızın sonuçlarını özetleyiniz. Sonuçlarınızın başlangıçta belirlediğiniz araştırma sorusuna ne derece cevap verdiğini ve ileride bu çalışmanın nasıl geliştirilebileceğini tartışınız.

\textbf{Kaynakça bölümü Rmarkdown tarafından otomatik olarak oluşturulmaktadır. Taslak dosyada Kaynakça kısmında herhangi bir değişikliğe gerek yoktur.}

\textbf{\emph{Taslakta bu cümleden sonra yer alan hiçbir şey silinmemelidir.}}

\newpage

\hypertarget{references}{%
\section{Kaynakça}\label{references}}

\hypertarget{refs}{}
\begin{CSLReferences}{0}{0}
\end{CSLReferences}

\end{document}
